\documentclass[sigconf]{acmart}

\usepackage{graphicx}
\usepackage{hyperref}
\usepackage{todonotes}

\usepackage{endfloat}
\renewcommand{\efloatseparator}{\mbox{}} % no new page between figures

\usepackage{booktabs} % For formal tables

\settopmatter{printacmref=false} % Removes citation information below abstract
\renewcommand\footnotetextcopyrightpermission[1]{} % removes footnote with conference information in first column
\pagestyle{plain} % removes running headers

\newcommand{\TODO}[1]{\todo[inline]{#1}}

\begin{document}
\title{Big Data And Data Visualization}


\author{Pravin Deshmukh}
\orcid{1234-5678-9012}
\affiliation{%
  \institution{Indiana University}
  \streetaddress{300 N. Jordan Avenue}
  \city{Bloomington} 
  \state{Indiana} 
  \postcode{47405-1106}
}
\email{praadesh@iu.edu}


\begin{abstract}
This article provides an overview on importance of data visualization in presenting findings of Big Data solutions. Data visualization is very important to understand Big Data analytic.
\end{abstract}

\keywords{Data Visualization, Big Data \LaTeX}


\maketitle
\section{Introduction}

Big data is widely used technology to consume huge amount of data. While there are various technologies available to process this data, for human is becomes very difficult to extract meaningful information when data becomes extremely large. Not many conventional  visualization tools are designed to present meaningful and quality information for human perception. Hence it becomes very important to have interactive, intuitive and user friendly data visualizations in place so that decision makers, business users will have clear understanding of findings of big data solutions. These visualizations will help business users to make informed decision looking at various trends over the period of time.

\section{Big Data}
Big data by definition refers to the any large quantity of raw data that can be collected, stored and analyzed through various means.Big Data is used for data sets that are so large and complex that traditional data processing tools are inadequate to store and process them. Big data challenges includes capturing data, data storage, data analysis, search, sharing, transfer, visualization, querying updating etc. 

Big data has become more relevant recently because of the data exponential data growth happened in last decade or so. 90\% of the data exists today is created in last 2 years\cite{sept1001}. Following are some interesting facts regarding this data explosion:
\begin{itemize}
    \item 2.7 Zetabytes of data exist in the digital universe today
    \item Facebook stores, accesses, and analyzes 30+ Petabytes of user generated data
    \item Brands and organizations on Facebook receive 34,722 Likes every minute of the day
    \item Walmart handles more than 1 million customer transactions every hour, which is imported into databases estimated to contain more than 2.5 petabytes of data
    \item More than 5 billion people are calling, texting, tweeting and browsing on mobile phones worldwide
    \item 571 new websites are created every minute of the day
\end{itemize}\cite{sept1002}

With this kind of data growth, it has become challenging for companies to process data to cleans data, identify good or bad data and produce meaningful outcome.

\section{Data Visualization}

Data Visualization is representation of the data in a visual context which helps users to easily understand the significance of the data. A primary goal of data visualization is to communicate information clearly and efficiently using visuals like statistical graphs, plots and information graphics\cite{sept1003}. Trends, patterns and correlation between entities which are not evident in in text-based data or numbers can be exposed and recognized with the help of data visualization. It makes complex statistics more accessible, understandable and usable\cite{sept1003}

Big data visualization is not as simple as it used to be with traditional smaller datasets.  In Big data data visualization, many data scientists use feature extraction and geometric modeling to drastically reduce data size before actual data rendering. Choosing proper data representation is also very important when visualizing big data.\cite{sept1004} 

\section{Challenges of Data Visualization}

Visualization of large data set is a quite demanding task. The conventional ways of presenting data have limitations because of the constant growth of the data. Modern Data visualization techniques have greatly evolved over the period of time. These modern visualization techniques and tools are designed to deal with following challenges posed by big data:

\begin{itemize}
    \item Volume : As mentioned above, rate of data growth has increased significantly in last few years. These tools are designed to process huge amount of data and allows users to derive meaningful outcome from very large amount of data   
    \item Velocity : With fast moving business dynamics, results are expected to be delivered real-time or near real time time frames hence tool are designed to process data in real time instead of batch processing 
    \item Variety : With the introduction of social media sites, requirement of processing variety of data has increased like never before. These tools are developed to integrate with wide variety of data sources which includes structured data, semi-structured data and non-structured data. 
\end{itemize}


\section{Data Visualization is the key to actionable insights}

There is a old saying  ``picture is worth a thousand words'' That's because an image can often convey "what's going on", more quickly, more efficiently, and often more effectively than words. Big data visualization techniques exploit this fact: they are all about turning data into pictures by presenting data in pictorial or graphical format This makes it easy for decision-makers to take in vast amounts of data at a glance to "see" what is going on

Visualization allows business to take complex findings and present them in a way that is informative and engaging to all stakeholders – and a strong understanding of data science is required for that visualization to be successful. We must all remember that in the end, the consumer of the product of all artificial intelligence or machine learning endeavors will be people. We should ensure results are delivered as actionable, impactful insights to act upon in business and in life. The human brain is only able to process two to three pieces of information at a time and many different aspects of consumer behavior are influenced by more than just two or three events. This means you have to utilize advanced analytics and statistical modeling to accurately predict consumer behavior and Key Performance Indicators (KPIs) for businesses.

Table1 shows benefits of data visualization 

\begin{center}
 \begin{tabular}{||c c c c||} 
 \hline
 Benefits & Percentage(\%) \\ [0.7ex] 
 \hline
 Improved decision-making & 77 \\ 
 \hline
 Better ad-hoc data analysis & 43  \\ 
 \hline
 Improved collaboration and information sharing & 41  \\ 
 \hline
 Provide self-service capabilities to business users & 36  \\ 
 \hline
 Improved return on investment (ROI) & 34  \\ 
 \hline
  Time savings & 20  \\ 
 \hline
  Reduced investment in IT & 15  \\ 
 \hline
\end{tabular}
\end{center}
\cite{sept1004}

\section{Data Visualization Tools}

Following are some popular and most widely used big data visualization tools :
\begin{itemize}
\item Cognos Analytics: Driven by their commitment to Big Data, IBM’s analytics package offers a variety of self service options to more easily identify insight.Cognos Analytics, an interactive way to search, explore, and share data-driven insights in a governed environment. Find precise and timely answers from your data or from content built by others. Create compelling reports and dashboards which you can easily distribute. Use automated alerts to monitor changes to key findings. 

\item QlikView: The Qlik solution proves its ability to perform the more complex analysis that finds hidden insights. Qlik data analytics platform can help companies gain the most leverage from a Big Data implementation by easing access and making Big Data both relevant and in-context for the organization's business users

\item Microsoft PowerBI: The Power BI tools enables you to connect with hundreds of data sources, then publish reports on the Web and across mobile devices.Power BI is a suite of business analytics tools that deliver insights throughout your organization. Connect to hundreds of data sources, simplify data prep, and drive ad hoc analysis. Produce beautiful reports, then publish them for your organization to consume on the web and across mobile devices. Everyone can create personalized dashboards with a unique, 360-degree view of their business. And scale across the enterprise\cite{sept1007}

\item Oracle Visual Analyzer: A web-based tool, Visual Analyzer allows creation of curated dashboards to help discover correlations and patterns in data. 

\item SAP Lumira: Calling it “self service data visualization for everyone,” Lumira allows you to combine your visualizations into storyboards. data visualization software that makes it easy to create beautiful and interactive maps, charts, and info-graphics. Import data from Excel and many other sources, perform visual BI analysis using intuitive dashboards, and securely share insights and data stories 

\item SAS Visual Analytics: The SAS solution promotes its ``capability and governance'' along with dynamic visuals and flexible deployment options. It provides single, powerful in-memory environment which allows Interactive reporting. Visual data discovery. Self-service analytics. Scalability and governance.

\item Tableau Desktop: Tableau’s interactive dashboards allow users to ``uncover hidden insights on the fly,'' and power users can manage metadata to make the most of disparate data sources.  Tableau as an organization has been dedicated to data visualization for over a decade and the results show in several areas: particularly in usability they have edge over their other competitors. Tableau’s functionality from an end-user perspective is much better than their closest rivals

\item TIBCO Spotfire: Offers analytics software as a service, and proves itself as a solution that scales from a small team to the entire organization. Spotfire represents the state of the art in visual data discovery, analytic applications and self service dashboard creation - and also also extends its capabilities into Mobile dashboards/apps
\end{itemize}\cite{sept1006}

\section{Conclusion}

 Today we have access to largest amount of data we ever had. But all data in the world is useless, in fact it will become liability, if you cannot understand it. Data visualization is all about presenting data to right people at right time.  As more and more businesses are analyzing their data with big data tools, data visualization is becoming an increasingly important component of analytics in the age of big data. The availability of new in-memory technology and high-performance analytics that use data visualization is providing a better way to analyze data more quickly than ever. Visual analytics enables organizations to take raw data and present it in a meaningful way that generates the most value. Nevertheless, when used with big data, visualization is bound to lead to some challenges. If we are prepared to deal with these hurdles, the opportunity for success with a data visualization strategy is much greater\cite{sept1005} 

\begin{acks}

  The authors would like to thank Prof. Gregor von Laszewski and entire team of TAs for all the help and support they provided during whole process of writing this paper.

\end{acks}



\bibliographystyle{ACM-Reference-Format}
\bibliography{report} 

\end{document}
