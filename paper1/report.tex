\documentclass[sigconf]{acmart}

\usepackage{hyperref}

\usepackage{endfloat}
\renewcommand{\efloatseparator}{\mbox{}} % no new page between figures

\usepackage{booktabs} % For formal tables

\settopmatter{printacmref=false} % Removes citation information below abstract
\renewcommand\footnotetextcopyrightpermission[1]{} % removes footnote with conference information in first column
\pagestyle{plain} % removes running headers

\begin{document}
\title{Big Data and Data Visualization}


\author{Pravin Deshmukh}
\orcid{1234-5678-9012}
\affiliation{%
  \institution{Indiana University}
  \streetaddress{P.O. Box 1212}
  \city{Bloomington} 
  \state{Indiana} 
  \postcode{43017-6221}
}
\email{praadesh@iu.edu}

% The default list of authors is too long for headers}
\renewcommand{\shortauthors}{Pravin Deshmukh}


\begin{abstract}
This paper will provide an overview on how analytical findings of Big Data solutions can be visualized using various visualization technologies 
\end{abstract}

\keywords{i523}


\maketitle

\section{Introduction}

Big data is widely used technology to consume huge amount of data. While there are various technologies available to process this data it is very important to have interactive, intuitive, user friendly data visualizations in place so that decision makers, business users will have clear understanding of findings of big data solutions. These visualizations will make help us to make informed decision looking at various trends over the period of time. 


\begin{acks}

  The authors would like to thank 

\end{acks}

\bibliographystyle{ACM-Reference-Format}
\bibliography{report} 

\end{document}
